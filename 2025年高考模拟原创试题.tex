\documentclass[fontset=windows]{article}
\usepackage[margin=1in]{geometry}%设置边距,符合Word设定
\usepackage{ctex}
\usepackage{setspace}
\usepackage{graphicx}%插入图片
\usepackage{amsmath}%矩阵符号
\usepackage{amssymb}
\usepackage{bm}

\usepackage{indentfirst} % 导入indentfirst包
\setlength{\parindent}{0em} % 设置首行缩进为0个字符宽度

\graphicspath{{Figures/}}%文章所用图片在当前目录下的 Figures目录


\title{\heiti\zihao{2} 2025年天津市高考模拟试题 }
\author{\songti }
\date{}
\setcounter{secnumdepth}{0}%不展示章节号

\begin{document}%正文从这里开始
	\maketitle
	\thispagestyle{empty}

\section{一、选择题}
\subsection{1}
集合$A=\{1,2,4,5\},B=\{2,3,4,5\}$,则$A\cup B=$\underline{\hbox to 1cm{}}。\\
A. $\{1,2,3,4,5\}$\quad B. $\{1,2,4,5\}$\quad C. $\{1,3,4,5\}$\quad D. $\{1,3,5\}$
\subsection{2}
已知$\forall x \in A, x \in B$则下列命题一定正确的是\underline{\hbox to 1cm{}}。\\
A. $\forall x \notin A, x \notin B$\qquad \qquad
B. $\exists x \notin A, x \in B$\\
C. $\forall x \notin B, x \notin A$\qquad \qquad
D. $\exists x \in B, x \notin A$
\subsection{3}
下列函数中,是奇函数的是\underline{\hbox to 1cm{}}。\\
A. $y=\ln\dfrac{x^2-1}{x^2+1}$\qquad \qquad
B. $y=\ln\dfrac{(x-1)^2}{(x+1)^2}$\\
C. $y=\ln\dfrac{x+2}{x-1}$\qquad \qquad ~
D. $y=\ln\dfrac{x-2}{x+1}$


\section{二、填空题}
\subsection{10}
已知$\mathrm{i}$为虚数单位,则$\left| \dfrac{3+4i}{25} \right|= $\underline{\hbox to 1cm{}}。
\subsection{11}
在$(x+\dfrac{1}{x})(x-\dfrac{x}{2})^3$的展开式中,所有系数之和为\underline{\hbox to 1cm{}}。
\subsection{13}
箱子里有6个小球,其中有3个蓝色小球,3个红色小球,现在从其中拿出小球,每次拿出两个,规定如果拿出的两个小球颜色相同则放入一个黄色小球,否则放入一个红色小球,则第三次取出两个黄色小球的概率为\underline{\hbox to 1cm{}}。反复取出小球,直到箱子里最后只剩下1个小球,则这个小球不可能是\underline{\hbox to 1cm{}}色小球。

\section{三、解答题}
\subsection{19}
数列$\{a_n\},\{b_n\}$满足$\{a_n\}$等差,$a_1=1,b_1=3,a_{n+1}=2a_{n-1}-a_n+6,b_{n+1}=(a_{n+1}-a_n)b_n$。\\
(1)求$\{a_n\},\{b_n\}$的通项公式;\\
(2)求$\displaystyle\sum\limits_{i=1}^{n}a_nb_n$。\\
(3)数列$\{c_n\}$满足$c_1=\dfrac{\sqrt3}2,
c_{n+1}=\sqrt{\dfrac{1-\sqrt{1-c_n^2}}{2}}$,证明:$b_nc_n<\dfrac{23}{7}$。
\subsection{20}
已知函数$f(x)=x^x,g(x)=x^\frac{1}{x}$\\
(1)求$f(x)$的最小值和$g(x)$的最大值.\\
(2)设$f(x)$和$g(x)$的交点为$C$,$l$为过$C$的一条直线,且$l$与$f(x),g(x)$的图象分别交于另两个点$A,B.$
.\qquad(i) 求$l$斜率的取值范围.\\
.\qquad(ii) 设$A,B$的横坐标分别为$x_1,x_2$,证明:$x_1x_2<1,x_1+x_2>2.$\\
\end{document}